\chapter{Discussion}


\texttt{eBayes} method is based on obtaining estimate for hyperparameters followed by MCMC to estimate gene-specific parameters. The empirical Bayes posteriors were used to estimate posterior probabilities of DE . Through a simulation study, we demonstrated that this method outperformed alternative methods in most simulation scenarios with higher AUC values. More samples (nSamples) would improve all methods' performance given the same proportion of DE genes (pDiff). DE analysis methods work better on count data with more replicates per condition and higher true DE genes proportion. This is not surprising considering that the focus of most methods is to model the variability in gene expression measurements and therefore increasing the number of replicates strengthen the estimate. The true DE genes proportion (pDiff) affects the outperformance of \texttt{eBayes}. When pDiff is smaller, \texttt{eBayes} performs much better than other methods in terms of AUC values, which means eBayes method has some advantages handling smaller true DE proportion scenario. We observed that the ROC curves for {\tt sSeq, DESeq} were almost straight lines in some senarios, which might be associated with equal cost of misclassifying DE and misclassifying non-DE cases. We got rid of such issues when we used unadjusted p-value as the statistics of {\tt sSeq, DESeq}. But using unadjusted p-values ended up inherently including a large number of false positives given cutoff of $0.05$ for p-values. Soneson \citep{soneson2013comparison} also mentioned that some methods including {\tt DESeq} exhibit an excess of large p-values, which has been attributed to the use of exact tests based on discrete probability distributions\citep{robles2012efficient}. 

For future researches, we would recommend adding more methods to the comparison, such as other Bayesian methods implemented by {\tt baySeq, ShrinkSeq}, nonparametric methods implemented by {\tt NOISeq, SAMseq} , and fully Bayesian method using {\tt fbseq}. The fully Bayesian method would involve some heavy computation. It would be easier with access to high performance clusters equipped with gpu nodes. People could also try more complicated experimental designs, i.e., including more experimental conditions or adding the flow cell effects. \texttt{eBayes} method might be improved by refining the hierarchical model for the gene-specific parameter distribution and the hyperparameters estimation. 