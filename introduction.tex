% Introduction of the Thesis Template File
\chapter{Overview}

\section{Introduction}

A gene is differentially expressed when the expected count reads of this gene corresponding to one variety differs from that of another variety. Finding genes that are differentially expressed between varieties is an integral part of understanding the molecular basis of phenotypic variation. High throughout sequencing of cDNA (RNA-Seq) has been used to quantify the abundance of mRNA corresponding to different genes\citep{soneson2013comparison}.

RNA-Seq is a new approach to transcriptome analysis based on next-generation sequencing technology. RNA-Seq data are a set of short RNA reads that are often summarized as discrete counts. The negative binomial distribution has become widely used to analyze RNA-Seq data which allows more flexibility in assigning between-sample variation\citep{ching2014power}. 

We analyzed simulated datasets with a defined total number of genes (nGenes), total number of samples (nSample), and proportion of DE genes (pDiff). We compared {\tt eBayes}\citep{niemi2015empirical} method to the alternative methods implemented by existing R packages, such as {\tt edgeR, DESeq, DESeq2, sSeq, EBSeq}. All methods are available within the R framework and take a matrix of counts as input, i.e., the number of reads mapping to each genomic feature of interest in each of a number of samples. We evaluated the methods based on simulated datasets, as demonstrated in RNA-Seq Data section. 


{\tt eBayes}\citep{niemi2015empirical} method was based on an approach to assess gene expression heterosis using microarray data under the assumption that these data are continuous, which was introduced by Ji et al. \citep{ji2014estimation}. Ji et al. built a normal hierarchical model for microarray measurements of transcript abundance that allows borrowing of information across genes to estimate means and variances. Ji et al. introduced an empirical Bayes framework that first estimates model hyperparameters, then estimates the posterior distribution for gene-specific parameters conditional on those hyperparameters, and finally computes heterosis probabilities based on integrals of regions under this posterior. Building on the work of Ji et al. with the normal data model, Niemi et al. \citep{niemi2015empirical} constructed a hierarchical model based on a negative binomial data model. Niemi et al. utilized an empirical Bayes approach to obtain estimates of the hyperparameters and the posterior distributions for the gene-specific parameters conditional on those hyperparameters. In this creative component report, we applied Niemi et al.’s empirical Bayes method in the differential gene expression (DE) analysis context, call it {\tt eBayes} method. 

Five alternative methods for differential expression (DE) analysis of RNA-Seq data were also evaluated in this study. All of them work on the count data directly: {\tt edgeR}\citep{robinson2010edger}, {\tt DESeq}\citep{anders2010differential}, {\tt DESeq2}\citep{love2014moderated}, {\tt sSeq}\citep{yu2013sseq}, {\tt EBSeq}\citep{leng2013ebseq}. More detailed descriptions of the methods can be found in the Method section and in the respective original publication. 

The six methods were evaluated based on simulated datasets, where we could control the settings and the true differential expression status of each gene. Details regarding the different simulation scenarios can be found in the Simulation section. We explored each method's ability to rank truly DE genes ahead of non-DE genes. This was evaluated in terms of the area under a Receiver Operating Characteristic (ROC) curve (AUC). 

The remainder of the report proceeds as follows. Chapter 2.1-2.3 presents the hierarchical model, the empirical Bayes method of estimating the parameters, and the calculation of posterior probabilities of DE. Chapter 2.4 presents the five alternative methods. Chapter 3 presents a simulation study based on a maize experiment and compares {\tt eBayes} approach to five alternative methods. Chapter 4 summarizes the result with a facetted AUC plot and includes the discussion part for future research direction. 



\section{RNA-Seq Data}

RNA-Seq is a next generation sequencing (NGS) procedure of the entire transcriptome by which one can measure the expression of several features, such as gene expression. The number of reads mapped to a given gene is considered to be the estimate of the expression level of that feature using the technology\citep{marioni2008rna}.

The end-product of a RNA-seq experiment id a sequence of read counts, typically a matrix with rows representing genes and columns representing samples from different gene varieties, as in Table \ref{tab:RNA-Seq Data}. In this example, there are $V=2$ gene varieties: $B73, Mo17$, $4$ replicates of each variety. The genes shown above the double horizontal line are part of the genes with differential expression between the two varieties. The genes shown below the double horizontal line are examples of the non-DE genes. 

\begin{table}[ht]
\centering
\begin{tabular}{|r|r|r|r|r|r|r|r|r|}
  \hline
 & B73\_1 & B73\_2 & B73\_3 & B73\_4 & Mo17\_1 & Mo17\_2 & Mo17\_3 & Mo17\_4 \\ 
  \hline
AC148152.3\_FG001 &   3 &   4 &   6 &   0 &   8 &  17 &  18 &  20 \\ 
  AC148152.3\_FG008 &   3 &   3 &   4 &   1 &  31 &  40 &  45 &  49 \\ 
  AC152495.1\_FG002 &  33 &  46 &  18 &  13 &   4 &   0 &   2 &   6 \\ 
  AC152495.1\_FG017 &  41 &  44 &  16 &  13 &   2 &   2 &   2 &   0 \\ 
  AC184130.4\_FG012 &  24 &  47 &  18 &  21 & 110 & 144 & 121 &  96 \\ 
  AC184133.3\_FG001 &   0 &   1 &   1 &   0 &  14 &  13 &   4 &   9 \\ 
   \hline
   \hline
AC148152.3\_FG005 & 2323 & 1533 & 1932 & 1945 & 2070 & 1582 & 2196 & 1882 \\ 
  AC148167.6\_FG001 & 672 & 598 & 728 & 713 & 743 & 655 & 821 & 824 \\ 
  AC149475.2\_FG002 & 459 & 438 & 451 & 483 & 467 & 448 & 634 & 532 \\ 
  AC149475.2\_FG003 & 1184 & 976 & 1131 & 1206 & 891 & 743 & 1288 & 1107 \\ 
  AC149475.2\_FG005 & 551 & 535 & 360 & 353 & 550 & 524 & 492 & 440 \\ 
  AC149475.2\_FG007 & 245 & 214 & 169 & 159 & 297 & 262 & 210 & 302 \\ 
   \hline
   
\end{tabular}
\caption{Maize RNA-Seq Data (Paschold,2012)}
\label{tab:RNA-Seq Data}
\end{table}

Our interest is in the detection of differentially expressed genes between the two varieties, i.e., genes for which read count distributions differ between varieties. 

