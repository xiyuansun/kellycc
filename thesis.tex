% Template file for a standard thesis
\documentclass[11pt]{isuthesis}
\usepackage[pdftex]{graphicx}
%\usepackage[utf8]{inputenc}
%\usepackage{float}
%\usepackage{graphicx}
%\usepackage[english]{babel}
\usepackage{amsmath}
\usepackage{float}

% Standard, old-style thesis
\usepackage{isutraditional}   \chaptertitle
\usepackage{comment}

% Old-style, thesis numbering down to subsubsection
\alternate
\usepackage{rotating}


% Bibliography without numbers or labels
\usepackage[round]{natbib}
\bibliographystyle{plainnat}


%Optional Package to add PDF bookmarks and hypertext links
\usepackage[pdftex,hypertexnames=false,linktocpage=true]{hyperref}
\hypersetup{
	colorlinks=true,
	linkcolor=blue,
	anchorcolor=blue,
	citecolor=blue,
	filecolor=blue,
	urlcolor=blue,
	bookmarksnumbered=true,
	pdfview=FitB
}

% Use import if there are .tex files called from chapter1.tex (a table for instance) 
\usepackage{import} 

% include path for figures
\graphicspath{{figures/}}

\newenvironment{thesis}{}{}
\excludecomment{thesis} % will exclude material that is only for the thesis


\begin{document}
\DeclareGraphicsExtensions{.jpg,.pdf,.mps,.png}
% Template Titlepage File
\title{Compare RNA-Seq Differential Expression Analysis Methods}

\author{Xiyuan Sun}
\mprof{Jarad Niemi}
\major{Statistics Department}

%--------- MASTER OF SCIENCE -------------
\degree{MASTER OF SCIENCE}
\level{master's}
\members{Jarad Niemi \\ Danniel Nettleton \\ Peng Liu}
%-----------------------------------------
\notice

%-------------  PhD Dissertation -------------------
% Add these additional lines for a Doctoral Dissertation
%\degree{Master of Science}
%\level{master}
\format{dissertation}
\committee{3}


%-----------CREATIVE COMPONENT ------------------------------
% Add these additional lines for a Creative Component
% - also comment out the \maketitle command
%\format{Creative Component}
%\submit{the graduate faculty}
\maketitle

% Optional thesis dedication
\chapter*{DEDICATION}

I would like to dedicate this thesis to ...


% Table of Contents, List of Tables and List of Figures
\pdfbookmark[1]{TABLE OF CONTENTS}{table}
\tableofcontents
\addtocontents{toc}{\def\protect\@chapapp{}} \cleardoublepage \phantomsection
\addcontentsline{toc}{chapter}{LIST OF TABLES}
\listoftables
\cleardoublepage \phantomsection \addcontentsline{toc}{chapter}{LIST OF FIGURES}
\listoffigures
% Comment out the next line if NOT using chaptertitle
\addtocontents{toc}{\def\protect\@chapapp{CHAPTER\ }}



\cleardoublepage \phantomsection This research was built on Niemi et al's approach \citep{niemi2015empirical}. Their research was supported by National Institute of General Medical Sciences (NIGMS) of the National Institutes of Health and joint National Science Foundation / NIGMS Mathematical Biology Program under award number R01GM109458.  
\cleardoublepage \phantomsection Differential gene expression (DGE) refers to the difference in the gene expression between two or more varieties based on read counts from replicated samples. An approach for comparing gene expression levels between replicated conditions of RNA sequencing data replies on counting reads that map to features of insterest. Within such count-based methods, many flexible and advanced statistical approaches now exist and offer the ability to adjust for covariates (e.g., batch effects) \cite{zhou2014robustly}. often, these methods include some sort of sharing of information across features to improve inferences in small samples. Jarad developed a hierarchical negative binomial model and drew inferences using a computationally tractable empirical Bayes approach to inference \cite{niemi2015empirical}. In this report, I demonstrate improvements over alternative methods via a simulation study based on a maize experiment. This article has supplementary material online \href{https://github.com/jarad/eBayes_differential_expression}{GitHub repo for this report}         



\newpage
\pagenumbering{arabic}

% Introduction of the Thesis Template File
\chapter{OVERVIEW}



\section{Introduction}

Differential expression exists when the expected value of a phenotype differs from the expected phenotypic values of the other variety. Differential expression can occue if the mean phenotype of a variety is greater than the other variety's or less than the other variety's. I refer to the former as high differential expression (HDE) and the latter as low differential expression (LDE). There is differential expression (DE) if and only if either HDE or LDE holds. Ji et al. \cite{ji2014estimation} introduced an approach to assess gene expression heterosis using microarray data under the assumption that these data are continuous. They built a normal hierarchical model for microarray measurements of transcript abundance that allows borrowing of information across genes to estimate means and variances. They introduced an empirical Bayes framework that first estimates model hyperparameters, then estimates the posterior distribution for gene-specific parameters conditional on those hyperparameters, and finally computes heterosis probabilities based on integrals of regions under this posterior. Building on the work of Ji et al. with the normal data model, Niemi et al. \cite{niemi2015empirical} constructed a hierarchical model based on a negative binomial data model. They utilized an empirical Bayes approach to obtain estimates of the hyperparameters and the posterior distributions for the gene-specific parameters conditional on those hyperparameters. In this report, I applied Niemi's empirical Bayes method in the differential gene expression analysis context. 


The remainder of the report proceeds as follows.Chapter 2 presents the hierarchical model, an empirical Bayes method of estimating the parameters, and the calculation of posterior probabilities of DE. Chapter 3 presents a simulation study based on a maize experiment and compares Niemi et al's approach to alternative methods. Chapter 4 summarizes the work and suggests directions for future research.


\section{RNA-Seq Count Data}
The focus here is on methods for count-based differential expression (DE) amalyses. Thus, the starting point here is a count table of features-by-samples, such as those from the Paschold project \cite{paschold2012complementation}.

The number of reads mapped to a given gene is considered to be the estimate of the expression level of that feature using the technology. The end product of a RNA-seq experiment is a sequence of read counts, typically represented as a matrix with rows representing genes and columns representing samples from two varieties, as in Table \ref{tab:RNA-Seq Data}. In this example, there are $V=2$ varieties, $J_1 = 2$ samples in the first variety, $J_2=2$ samples in the second variety, and $G=10000$ genes. My interest is in the detection of differentially expressed genes among the varieties. 

\begin{table}[H]
\begin{center}
    \begin{tabular}{|c|c|c|c|c|c|}
      \hline
      Gene &Variety1 Sample1 &Variety1 Sample2 &Variety2 Sample1 & Variety2 Sample2 \\
      \hline
      1 & 4 & 1 & 100 & 88 \\
      \hline
      2 & 65 & 48 & 55 & 59 \\
      \hline
      3 & 0 & 1 & 0 & 2\\
      \hline
      ... & ... & ... & ... & ...\\
      \hline
       10000 & 3 & 1 & 1 & 2\\
       \hline
    \end{tabular}
\end{center}
\caption{RNA-Seq Count Data Example}
\label{tab:RNA-Seq Data}
\end{table}

The focus of this report is to provide a comparison of the methods related to the analysis of differential expression for RNA-seq data. In this report, I review statistical methods for detecting differential expression in the RNA-seq data, including the empirical Bayesian method. I summarize the results of a simulation study. I briefly describe some existing open source R and Bioconductor software for testing differential expression for RNA-seq data. I conclude the report with a discussion section. 




\chapter{METHOD}

\section{Estimating the difference between read counts for a given gene}

To detemine whether the read count differences between different conditions for a given gene are greater than expected by chance, differential gene expression (DGE) tools must find a way to estimate that difference \citep{dundar2015introduction}.The two basic tasks of all DGE tools are: (1) Estimate the magnitude of differential expression between two or more conditions based on read counts from replicated samples, i.e., calculate the fold change of read counts, taking into account the differences in sequencing depth and variability; (2) Estimate the significance of the difference and correct for multiple testing. 

\section{Empirical Bayes identification of gene differential expression from RNA-seq read counts}

I consider an RNA sequencing (RNA-seq) experiment that involves two genetic varieties. For each variety, replicate RNA samples are isolated and assessed for quality. Complementary DNA (cDNA) libraries, consisting of short cDNA fragments derived from RNA, are constructed. Then, next generation sequencing technology is used to determine the {\tt reads}, in the cDNA libraries \citep{niemi2015empirical}. These reads are processed using bioinformatic algorithms to match the reads to genes. The results of read processing are summarized by a gene $\times$ sample matrix of counts. See Datta and Nettleton \citep{datta2014statistical} for more details on RNA-seq experiments and data from a statistical perspective, and see \citep{paschold2012complementation} for the biological background behind the use of RNA-seq to study gene expression differentiation. 

To use RNA-seq counts to identify genes displaing differential expression (DE), I built a hierarchical to borrow information across gene-variety means and across gene-specific overdispersion parameters, estimate the hyperparameters using an empirical Bayes procedure, and calculate empirical Bayes posterior probabilities for DE. 

\subsection{Hierarchical model for RNA-seq counts}

Let $Y_{gij}$ be the count for gene $g=1,2,..., G$, variety $i=1,2$, and replicate $j=1,2,3,...,n_i$.

I assume

\begin{equation}
\label{eq:1}
Y_{gij} \stackrel{ind}{\sim} NB(\exp(\lambda_{gi}+\gamma_{ij}), \exp(\psi_g))
\end{equation}

with mean $\mu_{gij} = \exp{(\lambda_{gi}+\gamma_{ij})}$ and dispersion $\phi_g = \exp{(\psi_g)}$

where $\gamma_{ij}$ terms are normalization factors that account for differencees in the thoroughness of sequencing from sample to sample. 

Following \citep{ji2014estimation}, we reparameterize the gene-variety mean structure into the genespecific average $\beta_{g1}$ and half-variety difference $\beta_{g2}$. For our differential expression study where number of varieties is 2, we let $i=1,2$ indicate the two varieties. The reparameterization is

\begin{equation}
\label{eq:2}
\beta_{g1} = \frac{\lambda_{g1}+\lambda_{g2}}{2}, \beta_{g2} = \frac{\lambda_{g1}-\lambda_{g2}}{2}
\end{equation}

We assume a hierarchical model for the gene-specific mean parameters and overdispersion parameters. Initially, we assume the variety averages, half-variety averages, and overdispersion parameters follow normal distributions

\begin{equation}
\label{eq:3}
\beta_{g1} \stackrel{ind}{\sim} N(\eta_{\beta_1}, \sigma^2_{\beta_1}), \beta_{g2} \stackrel{ind}{\sim} N(\eta_{\beta_2} , \sigma^2_{\beta_2}), \psi_g \stackrel{ind}{\sim} N(\eta_\psi, \sigma^2_\psi)
\end{equation}

The link function is \ref{eq:4}
\begin{equation}
\label{eq:4}
log(\mu_{gij}) = \lambda_{gi} + \gamma_{ij} = x_i^T \beta_g + log(N_{ij})
\end{equation}


\subsection{Empirical Bayes Method}

We categorize the parameters of the model into gene-specific parameters $\theta = (\theta_1, ..., \theta_G)$ where $\theta_g = (\beta_{g1}, \beta_{g2}, \psi_g)$, normalization factors $\gamma = (\gamma_{11}, ..., \gamma_{V n_V})$, and hyperparameters $\pi = (\eta, \sigma)$ where $\eta = (\eta_{\beta_1}, \eta_{\beta_2}, \eta_\psi)$ and $\sigma = (\sigma_{\beta_1}, \sigma_{\beta_2}, \sigma_\psi)$. We obtain estimates for the hyperparameters and then base gene-specific inference on the posterior conditional on these estimates \citep{niemi2015empirical}.

To obtain normalization factors $\hat{\gamma}$, we use the weighted trimmed mean of $M$ values (TMM). We use edgeR to obtain genewise dispersion estimates, $\hat{\psi}_g$, and the generalized linear model methods to obtain estimates for the remaining gene-specific parameters ($\hat{\beta}_{g1}, \hat{\beta}_{g2}$)\citep{robinson2010scaling}. Using $\hat{\theta}_g = (\hat{\beta}_{g1} , \hat{\beta}_{g2}, \hat{\psi}_g)$, we estimate hyperparameters for the location and scale parameters in the hierarchical model using a central method of moments approach. 

Conditional on the estimated normalization factors $\hat{\gamma}$ and hyperparameters $\hat{\pi}$, we perform a Bayesian analysis to re-estimate the gene-specific parameters and describe their uncertainty \citep{niemi2015empirical}. Equation \ref{eq:5} shows that conditional on $\hat{\gamma}$ and $\hat{\pi}$, the gene-specific parameters are independent and therefore conditional posterior inference across the genes can be parallelized. 

\begin{equation}
\label{eq:5}
\begin{split}
& p(\theta | y, \hat{\pi}, \hat{\gamma})  \propto \\ & \prod_{g=1}^{G} \left[ \prod_{i=1}^{2} \prod_{j=1}^{n_i} NB(y_{gij} ; \exp(\lambda_{gi} + \hat{\gamma}_{ij}), \exp(\psi_g)) N(\beta_{g1} ; \hat{\eta}_{\beta_1}, \hat{\sigma}^2_{\beta_1}) p(\beta_{g2} ; \hat{\eta}_{\beta_2}, \hat{\sigma}_{\beta_2}) N(\psi_g ; \hat{\eta}_{\psi}, \hat{\sigma}^2_{\psi})  \right]
\end{split}
\end{equation}

To perform the conditional posterior inference on the gene-specific parameters, we use the statistical software Stan \citep{stan2014stan} executed through the RStan interface \citep{team2016rstan}. Stan implements a Hamiltonion Monte Carlo \citep{neal2011mcmc} to obtain samples from the posterior in equation \ref{eq:5}. 


\subsection{Gene expression differentiation}

In the maize context that motivates this work, we are interested in differential expression (DE). For a specific gene $g$, non-DE occurs when expected expression in the second variety is the same as the expected expression of first variety, i.e., $\mu_{g1} = \mu_{g2}$, or equivalently, $\beta_{g2}=0$.  I evaluate measurements based on empirical Bayes estimates of their posterior probabilities, e.g., 

\begin{equation}
\label{eq:6}
P(DE_g | y, \hat{\pi}, \hat{\gamma}) =min( P(\beta_{g2}< 0 | y, \hat{\pi}, \hat{\gamma}),  P(\beta_{g2}> 0 | y, \hat{\pi}, \hat{\gamma}))
\end{equation}

$$P(\beta_{g2}< 0 | y, \hat{\pi}, \hat{\gamma}) \approx \frac{1}{M} \sum_{m=1}^M I(\beta_{g2} ^ {(m)} < 0)$$

$$P(\beta_{g2}> 0 | y, \hat{\pi}, \hat{\gamma}) \approx \frac{1}{M} \sum_{m=1}^M I(\beta_{g2} ^ {(m)} > 0) $$

where $\beta_{g2}^{(m)}$ is the $m^{th}$ MCMC sample from the empirical Bayes posterior.



I construct a ranked list of genes according to the minimum of $P(\beta_{g2}< 0 | y, \hat{\pi}, \hat{\gamma})$ and $P(\beta_{g2}> 0 | y, \hat{\pi}, \hat{\gamma})$. Geneticists can use this list to prioritize future experiments to understand the molecular genetic mechanisms for differential expression \citep{niemi2015empirical}. 

I will use the term ebayes to refer to the approach defined in Sections 2.1 - 2.2 and we are assuming normal distribution for half-variety differences.


\section{Simulation study based on a maize experiment}

To assess the efficacy of ebayes method to identify genes demonstrating DE, I used a maize dataset with two varieties B73 and Mo17\citep{paschold2012complementation} to determine realistic parameter values for a simulation study.I compared ebayes method to approaches using the R packages {\tt sSeq, DESeq, edgeR, EBSeq, and DESeq2}.

I used ebayes method to obtain normalization factors $\hat{\gamma}$ and gene-specific parameter estimates $\hat{\theta}_g$ for all genes using the {\tt edgeR} package\citep{robinson2010edger} applied to the maize data\citep{paschold2012complementation} on $27619$ genes with average count at least one and at most two zeros read counts for each variety across all the replicates. This analysis produced sample-specific normalization factors of $\hat{\gamma} = (0.923, 0.932, 1.029, 1.022, 0.973, 0.972, 1.063, 1.098)$. The gene-specific parameter estimates were treated as the true parameter values for the simulation study so that my simulated datasets mimicked the existing structure among the gene-variety means of the original maize data.

Using these parameters and normalization factors, I simulated data according to the negative binomial model in equation \ref{eq:1} independently for each gene, where $\exp{\psi_g}$ is the genewise dispersion calculated by Cox and Reid adjusted profile likelihood (CR-APL) method, the normalized factors $\gamma_{ij} = log(N_{ij})$, link function $\lambda_{gi} = X\beta_{gi}$, and the expected value $\mu_{gij} = \exp{(\lambda_{gi}+\gamma_{ij})} = N_{ij} \times \exp{(X\beta_{gi})}$. In the GLM setting, the mean response $\mu_{gij}$ is linked to a linear predictor with the base $e$ logarithm link according to equation \ref{eq:7}

\begin{equation}
\label{eq:7}
log(\mu_{gij}) = \lambda_{gi}+\gamma_{ij} = X\beta_{gi} + log(N_{ij})
\end{equation}

where $X$ is the design matrix containing the covariates (varieties), $\beta_g$ is a vector of regression parameters, and $N_{ij}$ is the effective library size for variety $i$ replicate $j$.

For each simulation, I analyzed a subset of $nGenes=1000$ or $nGenes=10000$ selected randomly from original maize count data. The library size of each sample in a simulated dataset was generated from a uniform distribution whose parameters were estimated from the maximum and minimum of the original maize count data. I repeated the simulation 5 times for each of 2, 4, and 8 replicates per variety, each of $1\%, 10\%, 30\%$ differentiation expression proportions, reusing normalization factors. 

Given the parameters gene-specific parameter estimates $\hat{\theta}_g$, I set up the simulation scenarios as the following table \ref{tab:Scenario}:

\begin{table}[H]
\centering
\begin{tabular}{|r|r|r|r|r|}
\hline
sc & nGenes & nSamples & pDiff \\ 
\hline
1 & 10000 & 8 & 0.10 \\ 
\hline
2 & 10000 & 8 & 0.30 \\ 
\hline
3 & 10000 & 8 & 0.01 \\
\hline
4 & 10000 & 4 & 0.10 \\
\hline
5 & 10000 & 4 & 0.30 \\
\hline
6 & 10000 & 4 & 0.01 \\ 
\hline
7 & 10000 & 16 & 0.10 \\
\hline
8 & 10000 & 16 & 0.30 \\ 
\hline
9 & 10000 & 16 & 0.01 \\
\hline
10& 1000 & 8 & 0.10 \\
\hline
11 & 1000 & 8 & 0.30 \\
\hline
12 & 1000 & 8 & 0.01 \\ 
\hline
13 & 1000 & 4 & 0.10 \\
\hline
14 & 1000 & 4 & 0.30 \\
\hline
15 & 1000 & 4 & 0.01 \\ 
\hline
16 & 1000 & 16 & 0.10 \\
\hline
17 & 1000 & 16 & 0.30 \\ 
\hline
18 & 1000 & 16 & 0.01 \\ 
\hline
\end{tabular}
\caption{Simulation Scenario Table}
\label{tab:Scenario}
\end{table}

For a particular gene, the truth was determined via the proportion of differential gene expression (pDiff) in the scenario setup. I randomly selected $nGenes \times (1-pDiff)$ genes and assign the expected variety count means the same as $\mu_{g1} = \mu_{g2} = 1/(n_1+n_2)\times \sum_{i=1}^2 \sum_{j=1}^{n_i} Y_{gij}$ for the selected non-DE genes. 



\section{Alternative DGE tools}

I compared the ebayes method to five alternative methods. To follow the recent progress in the RNA-Seq DE area, I selected two widely used methods, {\tt edgeR, DESeq}, and three other newly released DE analysis packages {\tt DESeq2, EBSeq, and sSeq}. For each method, I attempted to provide a measure of the strength of DE for each gene such that small values of this measure indicate support for DE. 

Among the alternative methods, {\tt sSeq}\citep{yu2013sseq}, {\tt DESeq}\citep{anders2010differential}, {\tt edgeR}\citep{robinson2007moderated}\citep{robinson2010edger}, {\tt EBSeq}\citep{leng2013ebseq}, and {\tt DESeq2}\citep{love2014moderated} assume the Negative Binomial distribution.


{\tt edgeR} is designed for the analysis of replicated count based expresison data and is an implementation of methodology developed by Robinson and Smyth \citep{robinson2007moderated}.{\tt edgeR} determines DE using empirical Bayes estimation and exact tests based on a negative binomial model. In particular, an empirical Bayes procedure is used to moderate the degree of overdispersion across genes by borrowing information between genes.An exact test analogous to Fisher's exact test but adapted to overdispersed data is used to assess DE for each gene. As default, the TMM normalization procedure is carried out to account for the different sequencing depths between the samples, wheareas the Benjamini-Hochberg procedure is used to control the FDR\citep{seyednasrollah2013comparison}. To construct a measure of DE, I computed the maximum likelihood estimates of the $\mu_{gi}$ parameters for all genes using edgeR's built-in Fisher scoring algorithm, and then used exact tests to calculate p-value for each gene $p_g$ for testing $H_{g0}: \mu_{g1} = \mu_{g2}$, which are adjusted for multiplicities using Benjamini-Hochberg. I use $p_g$ as a measure negatively associated with strength of evidence for DE. {\tt edgeR} moderates the dispersion estimate for each gene toward a local estimate from genes with similar expression strength, using a weighted conditional strength.

{\tt DESeq} also models the count of reads with the Negative Binomial distribution, but modeles the observed relationship between mean and variance when estimating dispersion, allowing a more general, data-driven parameter estimation\citep{seyednasrollah2013comparison}. It has the same hypothesis statement and exact test as {\tt edgeR}. {\tt DESeq} detects and corrects dispersion estimates that are too low through modeling of the dependence of the dispersion on the average expression strength over all samples. 

{\tt DESeq2} is a successor to {\tt DESeq}.{\tt DESeq2} sequentially estimates a prior distribution for the true dispersion values around the fit, then provide the maximum a posteriori (MAP) as the final estimate. It differs from {\tt DESeq}, which used the maximum of the fitted curve and the gene-wise dispersion estimate as the final estimate. It also differs from {\tt edgeR}, as {\tt DESeq2} estimates the width of the prior distribution from the data and therefore automatically controls the amount of shrinkage based on the observed properties of the data. In contrast, {\tt edgeR} require a user-adjustable parameter, the prior degrees of freedom, which weighs the contribution of the individual gene estimate and dispersion fit\citep{love2014moderated}. 


{\tt sSeq} can be used to test for differential expression between any two varieties based on the shrinkage estimation of dispersion in Negative Binomial models\citep{yu2013sseq}.The model has little practical difference from the model in {\tt DESeq}. Yu and Huber use the Hansen's generalized shrinkage estimator $\hat{\phi}_g$ in conjunction with the NB distribution to test genes for differential expression. They follow {\tt edgeR, DESeq} by testing $H_{g0}: \mu_{g1} = \mu_{g2}$ per gene with the exact test. Under $H_{g0}$, the p-values are calculated with respect to $Y_{gi.} \stackrel{H_{g0}}{\sim} NB(\sum_{j} s_{ij}\mu_{g}, \phi_{g}/\sum_{j} s_{ij})$ and are adjusted to control the false discovery rate\citep{yu2013sseq}. 

{\tt EBSeq} provides posterior probabilities as the evidence in favor of DE. Estimates of the gene-specific means and variances are obtained via method-of-moments, and the hyperparameters are obtained via the expectation-maximization (EM) algorithm\citep{leng2013ebseq}. {\tt EBSeq} estimates the posterior likelihoods of differential expression by the aid of empirical Bayesian methods. To account for the different sequencing depths, a median normalization procedure similar to {\tt DESeq} is used. 


\chapter{RESULT}

\section{Methods Comparison Result}

For the methods in Chapter 2, I sorted genes according to the computed measure of the strength of evidence for DE. From these sorted lists, I calculated area under ROC curve (AUC) values to evaluate the ability of these methods to distinguish genes with DE. 

Figure \ref{auc} provides the area under the ROC curve (AUC) across the 5 simulations for each of the scenario defined in \ref{tab:Scenario}. I facetted the plots by number of samples (nSample) and differential gene expression proportion (pDiff), grouped by different level of total number of genes. 

\begin{figure}[h!tb] 
\includegraphics[scale=0.4]{auc_facet_plot}
\caption{Facet AUC Plot}
\label{auc}
\end{figure}

In terms of AUC values, ebayes method always have promising DE analysis performance.

When number of samples increases, AUC values of all methods increase. 

When DE proportion increases, the difference among the methods decreases. 

There is no obvious difference when number of genes differ.


%\appendixtitle 
%\appendix
%\include{appendix1}
%\include{appendix2}


\renewcommand{\bibname}{\centerline{BIBLIOGRAPHY}}
\unappendixtitle
\newpage
\phantomsection
%\addcontentsline{toc}{chapter}{BIBLIOGRAPHY}
\bibliographystyle{plain}
\bibliography{mybib}

\end{document}
