\chapter{Method}

\section{Estimating the difference between read counts for a given gene}

To detemine whether the read count differences between different conditions for a given gene are greater than expected by chance, differential gene expression (DGE) tools must find a way to estimate that difference \citep{dundar2015introduction}.The two basic tasks of all DGE tools are: (1) Estimate the magnitude of differential expression between two or more conditions based on read counts from replicated samples, i.e., calculate the fold change of read counts, taking into account the differences in sequencing depth and variability; (2) Estimate the significance of the difference and correct for multiple testing. 

\section{Standard Setup of Negative Binomial Model in Generalized Linear Model Framework}

To most easily explain the following DE analysis methods, we followed closedly the notation used in McCarthy et al.'s paper\citep{mccarthy2012differential}.

Let the $Y_{gij}$ be the read cound in replicate $j$ of condition $i$ for gene $g$. Assume $y_{gij}$ follows a NB distribution with mean $\mu_{gij}$ and gene-wise dispersion $\phi_g$, denoted by $Y_{gij} \stackrel{ind}{\sim} NB(\mu_{gij}, \phi_g)$. Gene $g$'s varince equals $\mu_{gij}(1+\phi_g \cdot \mu_{gij})$, while the dispersion $\phi_g$ is the square of the biological coefficient of variation \citep{mccarthy2012differential}. 

In the generalized linear model (GLM) setting, the mean response, $\mu_{gij}$, is linked to a  linear predictor with the natural logorithm link according to $\log(\mu_{gij}) = \mathbf{X\beta_g} + \log(N_{ij})$, where $\mathbf{X}$ is the design matrix containing the covariates (e.g., experimental conditions, batch effects, etc.), $\mathbf{\beta_g}$ is a vector of regression parameters (a subset of which are of interest for differential expression inference) and $N_{ij}$ is the normalized library size for replicate $j$ of condition $i$. Different DE analysis methods adopted different algorithms to estimate the regression parameters and gene-wise dispersion parameters. More detials could be checked in the Alternative Methods section. 


\section{Empirical Bayes identification of gene differential expression from RNA-seq read counts}


To use RNA-seq counts to identify genes displaing differential expression (DE), we built a hierarchical model to borrow information across gene-variety means and across gene-specific overdispersion parameters, estimate the hyperparameters using an empirical Bayes procedure, and calculate empirical Bayes posterior probabilities for DE. 

\subsection{Hierarchical model for RNA-seq counts}

Let $Y_{gij}$ be the count for gene $g=1,2,..., G$, variety $i=1,2$, and replicate $j=1,2,3,...,n_i$.

We assume

\begin{equation}
\label{eq:1}
Y_{gij} \stackrel{ind}{\sim} NB(\mu_{gi}, \phi_g)
\end{equation}

Mapped to the notation system Niemi used in their paper\citep{niemi2015empirical}, we have the link functions as equation \ref{eq:2}
\begin{equation}
\label{eq:2}
\log(\mu_{gij}) = \lambda_{gi} + \gamma_{ij} = x_i^T \beta_g + \log_e (N_{ij})
\end{equation}
and dispersion $\phi_g = \exp{(\psi_g)}$, where $\gamma_{ij}$ terms are normalization factors that account for differencees in the thoroughness of sequencing from sample to sample. 

Following \citep{ji2014estimation}, we reparameterize the gene-variety mean structure into the genespecific average $\beta_{g1}$ and half-variety difference $\beta_{g2}$. For our differential expression study where number of varieties is 2, let $i=1,2$ indicate the two varieties. The reparameterization is

\begin{equation}
\label{eq:3}
\beta_{g1} = \frac{\mu_{g1}+\mu_{g2}}{2}, \beta_{g2} = \frac{\mu_{g1}-\mu_{g2}}{2}
\end{equation}

We assume a hierarchical model for the gene-specific mean parameters and overdispersion parameters. Initially, we assume the variety averages, half-variety averages, and overdispersion parameters follow normal distributions

\begin{equation}
\label{eq:4}
\beta_{g1} \stackrel{ind}{\sim} N(\eta_{\beta_1}, \sigma^2_{\beta_1}), \beta_{g2} \stackrel{ind}{\sim} N(\eta_{\beta_2} , \sigma^2_{\beta_2}), \psi_g \stackrel{ind}{\sim} N(\eta_\psi, \sigma^2_\psi)
\end{equation}


\subsection{Empirical Bayes Method (eBayes)}

We categorized the parameters of the model into gene-specific parameters $\theta = (\theta_1, ..., \theta_G)$ where $\theta_g = (\beta_{g1}, \beta_{g2}, \psi_g)$, normalization factors $\gamma = (\gamma_{11}, ..., \gamma_{V n_V})$, and hyperparameters $\pi = (\eta, \sigma)$ where $\eta = (\eta_{\beta_1}, \eta_{\beta_2}, \eta_\psi)$ and $\sigma = (\sigma_{\beta_1}, \sigma_{\beta_2}, \sigma_\psi)$. We obtained estimates for the hyperparameters and then based gene-specific inference on the posterior conditional on these estimates \citep{niemi2015empirical}.

To obtain normalization factors $\hat{\gamma}$, we used the weighted trimmed mean of $M$ values (TMM). We used {\tt edgeR} to obtain genewise dispersion estimates, $\hat{\psi}_g$, and the generalized linear model methods to obtain estimates for the remaining gene-specific parameters ($\hat{\beta}_{g1}, \hat{\beta}_{g2}$)\citep{robinson2010scaling}. Using $\hat{\theta}_g = (\hat{\beta}_{g1} , \hat{\beta}_{g2}, \hat{\psi}_g)$, I estimate hyperparameters for the location and scale parameters in the hierarchical model using a central method of moments approach. 

Conditional on the estimated normalization factors $\hat{\gamma}$ and hyperparameters $\hat{\pi}$, I perform a Bayesian analysis to re-estimate the gene-specific parameters and describe their uncertainty \citep{niemi2015empirical}. Equation \ref{eq:5} shows that conditional on $\hat{\gamma}$ and $\hat{\pi}$, the gene-specific parameters are independent and therefore conditional posterior inference across the genes can be parallelized. 

\begin{equation}
\label{eq:5}
\begin{split}
& p(\theta | y, \hat{\pi}, \hat{\gamma})  \propto \\ & \prod_{g=1}^{G} \left[ \prod_{i=1}^{2} \prod_{j=1}^{n_i} NB(y_{gij} ; \mu_{gij}, \phi_g) N(\beta_{g1} ; \hat{\eta}_{\beta_1}, \hat{\sigma}^2_{\beta_1}) p(\beta_{g2} ; \hat{\eta}_{\beta_2}, \hat{\sigma}_{\beta_2}) N(\psi_g ; \hat{\eta}_{\psi}, \hat{\sigma}^2_{\psi})  \right]
\end{split}
\end{equation}

To perform the conditional posterior inference on the gene-specific parameters, we used the statistical software Stan \citep{stan2014stan} executed through the RStan interface \citep{team2016rstan}. Stan implements a Hamiltonion Monte Carlo \citep{neal2011mcmc} to obtain samples from the posterior in equation \ref{eq:5}. We used the default NUTS sampler\citep{annis2017bayesian}.


\subsection{Gene expression differentiation}

In the maize context that motivates this work, we are interested in differential expression (DE). For a specific gene $g$, non-DE occurs when expected expression in the second variety is the same as the expected expression of first variety, i.e., $\mu_{g1} = \mu_{g2}$, or equivalently, $\beta_{g2}=0$.  I evaluate measurements based on empirical Bayes estimates of their posterior probabilities, e.g., 

\begin{equation}
\label{eq:6}
P(DE_g | y, \hat{\pi}, \hat{\gamma}) =\min( P(\beta_{g2}< 0 | y, \hat{\pi}, \hat{\gamma}),  P(\beta_{g2}> 0 | y, \hat{\pi}, \hat{\gamma}))
\end{equation}

$P(\beta_{g2}< 0 | y, \hat{\pi}, \hat{\gamma}) \approx \frac{1}{M} \sum_{m=1}^M I(\beta_{g2} ^ {(m)} < 0)$, $P(\beta_{g2}> 0 | y, \hat{\pi}, \hat{\gamma}) \approx \frac{1}{M} \sum_{m=1}^M I(\beta_{g2} ^ {(m)} > 0) $
where $\beta_{g1}^{(m)}, \beta_{g2}^{(m)}$ is the $m^{th}$ MCMC sample from the empirical Bayes posterior.


We constructed a ranked list of genes according to the minimum of $P(\beta_{g2}< 0 | y, \hat{\pi}, \hat{\gamma})$ and $P(\beta_{g2}> 0 | y, \hat{\pi}, \hat{\gamma})$. Geneticists can use this list to prioritize future experiments to understand the molecular genetic mechanisms for differential expression \citep{niemi2015empirical}. 

We will use the term {\tt eBayes} to refer to the approach defined in Sections 2.1 - 2.2 and we are assuming normal distribution for half-variety differences.

\section{Alternative Methods}

We compared the eBayes method to five alternative methods. To follow the recent progress in the RNA-Seq DE area, we selected two widely used methods, {\tt edgeR, DESeq}, and three other newly released DE analysis packages {\tt DESeq2, EBSeq, and sSeq}. For each method, we attempted to provide a measure of the strength of DE for each gene such that small values of this measure indicate support for DE. 

Several authors proposed differential expression methods based on the negative binomial distribution, motivated by observation that real RNA-Seq data sets typically exhibited greater variability than could be modeled via the Poisson distribution\citep{lorenz2014using}. 

Robinson and Smyth \citep{robinson2007moderated} assumed a negative binomial distribution for the read counts for all genes with a common dispersion parameter, i.e., $Y_{gij} \stackrel{ind}{\sim} NB(\mu_{gij},\phi)$, where $\mu_{gij}=N_{ij}\exp(\lambda_{gi})$, $N_{ij}$ is the library size for sample $j$ in population $i$, and $\exp(\lambda_{gi})$ is the relative abundance parameter for gene $g$ in population $i$, which is assumed to be the same to the replicate samples within a population. The dispersion parameter $\phi$ is estimated by maximizing the conditional likelihood given the sum of the counts in each poopulation. Quantile adjusted conditional maximum likelihood (qCML) is applied if the library sizes are not equal within each population. The null hypothesis for the test of differential expression is the equality of the relative abindance parameters, $H_0: \lambda_{g1} = \lambda_{g2}, g=1,2,...,G$. The authors suggested an exact NB test based on the same quantile adjustment used in estimating the dispersion parameter, and a p-value calculated as the probability of observing counts greater than those observed\citep{lorenz2014using}. But the assumption of dispersion parameter $\phi$ common to all genes is often implausible. The authors extended their NB approach and suggested using gene-specific dispersion parameter $\phi_g$\citep{robinson2007small}. The exact test with empirical Bayes adjustment was better at detecting DE genes and was better able to control false discovery rates when gene-specific overdispersion was introduced\citep{lorenz2014using}. So they extended the standard NB approach by estimating gene-specific dispersion parameter via empirial Bayes weighted likelihood estimation, in which gene-specific dispersion parameter estimates were shrunk toward a common dispersion. Their method was implemented in R package called {\tt edgeR}. It moderates the dispersion per gene toward a local estimate with genes of similar expression. 


Anders and Huber \citep{anders2010differential} noted that dispersion often varies with expected read count, and suggested an extended NB model in which the variances of the read count are defined a nonparametric function of their expectations, as $Y_{gij} \stackrel{ind}{\sim} NB(\mu_{gij}, \phi_\mu)$, where $\mu_{gij}=N_{ij}\exp(\lambda_{gi})$. Then $Var(Y_{gij}) = \mu_{gij}(1+\phi_\mu \mu_{gij})$. They employ a gamma-family generalized linear local regression to model the mean-dispersion relationship. The nulll hypothesis in the test of differential expression is $H_0: \lambda_{g1} = \lambda_{g2}$, which is tested by an exact test similar to Robinson and Smyth's. Their method was implemented in an R package called {\tt DESeq}. It detects and corrects dispersion estimates that are too low through modeling of the dependence of the dispersion on the average expression strength over all samples. {\tt DESeq} (by default) estimates dispersion by pooling all samples together, fitting them to a parametric distribution and taking the maximum.

Love and Huber \citep{love2014moderated} then proposed another method for differential analysis of count data, using shrinkage estimation for dispersions and fold change to improve stability and interpretability of the estimates based on Anders and Huber's. They noticed the limitation of the most common approach in the comparative analysis of transcriptomics data. The noisiness of LFC estimates for genes with low counts would complicate the ranking by fold change. So they developed a statistical framework to facilitate gene ranking and visualization based on stable estimation of effect sizes (LFCs), as well as testing of differential expression with respect to user-defined thresholds of biological significance. They first perform ordinary GLM fits to obtain MLEs for the LFCs and then fit a zero-centered normal distribution to the observed distribution of MLEs over all genes. This distribution is used as a prior on LFCs in a second round of GLM fits, and the MAP estimates are kept as the final estimates of LFC. A standard error for each estimate is derived from the posterior's curvature at its maximum. These shrunken LFCs and their standard errors are used in the Wald tests for differential expression. Their method was implemented by {\tt DESeq2}. It uses a Wald test: the shrinken estimate of LFC is divided by its standard error, resulting in a z-statistic, which is compared to a standard normal distribution. {\tt DESeq2} is a new update to {\tt DESeq}, and it uses shrinkage estimation for dispersion: the first round of dispersion-mean relationship is obtained by MLE, and this fit is then used as a prior to estimate the maximum a posteriori estimate for dispersion in the second round. 


Yu and Huber used the method of moment estimates for the dispersion and shrank them towards an estimated target, which minimizes the average squared difference between the shrinkage estimates an the initial estimates. They estimate dispersion by pooling all the samples using the method of moments, and then shrinking the gene-wise estimates through minimizing the mean-square error. They also used exact test for the DE analysis. The model has little practical difference from the model in Anders and Huber's. Yu and Huber use the Hansen's generalized shrinkage estimator $\hat{\phi}_g$ in conjunction with the NB distribution to test genes for differential expression. They follow {\tt edgeR, DESeq} by testing $H_{0}: \mu_{g1} = \mu_{g2}$ per gene with the exact test. Under $H_{0}$, the p-values are calculated with respect to $Y_{gij} {\sim} NB(s_{ij}\mu_{gi}, \phi_{gi})$ and are adjusted to control the false discovery rate\citep{yu2013sseq}. $s_{ij}$ is the size factor. It can be thought of the representative ratio of counts in the library to the geometric mean of the counts in all the libraries. Their method was implemented in {\tt sSeq}.


Leng developed an empirical Bayes model for identifying DE genes and isoforms. This method was implemented in {\tt EBSeq}. It provides posterior probabilities as the evidence in favor of DE. Estimates of the gene-specific means and variances are obtained via method-of-moments, and the hyperparameters are obtained via the expectation-maximization (EM) algorithm\citep{leng2013ebseq}. {\tt EBSeq} estimates the posterior likelihoods of differential expression by the aid of empirical Bayesian methods. To account for the different sequencing depths, a median normalization procedure similar to {\tt DESeq} is used. 