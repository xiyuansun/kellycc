\chapter{Mehod}

\section{Estimating the difference between read counts for a given gene}

To detemine whether the read count differences between different conditions for a given gene are greater than expected by chance, differential gene expression (DGE) tools must find a way to estimate that difference \citep{dundar2015introduction}.The two basic tasks of all DGE tools are: (1) Estimate the magnitude of differential expression between two or more conditions based on read counts from replicated samples, i.e., calculate the fold change of read counts, taking into account the differences in sequencing depth and variability; (2) Estimate the significance of the difference and correct for multiple testing. 

\section{Empirical Bayes identification of gene differential expression from RNA-seq read counts}

I consider an RNA sequencing (RNA-seq) experiment that involves two genetic varieties. For each variety, replicate RNA samples are isolated and assessed for quality. Complementary DNA (cDNA) libraries, consisting of short cDNA fragments derived from RNA, are constructed. Then, next generation sequencing technology is used to determine the {\tt reads}, in the cDNA libraries \citep{niemi2015empirical}. These reads are processed using bioinformatic algorithms to match the reads to genes. The results of read processing are summarized by a gene $\times$ sample matrix of counts. See Datta and Nettleton \citep{datta2014statistical} for more details on RNA-seq experiments and data from a statistical perspective, and see \citep{paschold2012complementation} for the biological background behind the use of RNA-seq to study gene expression differentiation. 

To use RNA-seq counts to identify genes displaing differential expression (DE), I built a hierarchical to borrow information across gene-variety means and across gene-specific overdispersion parameters, estimate the hyperparameters using an empirical Bayes procedure, and calculate empirical Bayes posterior probabilities for DE. 

\subsection{Hierarchical model for RNA-seq counts}

Let $Y_{gij}$ be the count for gene $g=1,2,..., G$, variety $i=1,2$, and replicate $j=1,2,3,...,n_i$.

I assume

\begin{equation}
\label{eq:1}
Y_{gij} \stackrel{ind}{\sim} NB(\exp(\lambda_{gi}+\gamma_{ij}), \exp(\psi_g))
\end{equation}

with mean $\mu_{gij} = \exp{(\lambda_{gi}+\gamma_{ij})}$ and dispersion $\phi_g = \exp{(\psi_g)}$

where $\gamma_{ij}$ terms are normalization factors that account for differencees in the thoroughness of sequencing from sample to sample. 

Following \citep{ji2014estimation}, I reparameterize the gene-variety mean structure into the genespecific average $\beta_{g1}$ and half-variety difference $\beta_{g2}$. For my differential expression study where number of varieties is 2, I let $i=1,2$ indicate the two varieties. The reparameterization is

\begin{equation}
\label{eq:2}
\beta_{g1} = \frac{\lambda_{g1}+\lambda_{g2}}{2}, \beta_{g2} = \frac{\lambda_{g1}-\lambda_{g2}}{2}
\end{equation}

I assume a hierarchical model for the gene-specific mean parameters and overdispersion parameters. Initially, I assume the variety averages, half-variety averages, and overdispersion parameters follow normal distributions

\begin{equation}
\label{eq:3}
\beta_{g1} \stackrel{ind}{\sim} N(\eta_{\beta_1}, \sigma^2_{\beta_1}), \beta_{g2} \stackrel{ind}{\sim} N(\eta_{\beta_2} , \sigma^2_{\beta_2}), \psi_g \stackrel{ind}{\sim} N(\eta_\psi, \sigma^2_\psi)
\end{equation}

The link function is \ref{eq:4}
\begin{equation}
\label{eq:4}
\log(\mu_{gij}) = \lambda_{gi} + \gamma_{ij} = x_i^T \beta_g + \log_e (N_{ij})
\end{equation}


\subsection{Empirical Bayes Method}

I categorize the parameters of the model into gene-specific parameters $\theta = (\theta_1, ..., \theta_G)$ where $\theta_g = (\beta_{g1}, \beta_{g2}, \psi_g)$, normalization factors $\gamma = (\gamma_{11}, ..., \gamma_{V n_V})$, and hyperparameters $\pi = (\eta, \sigma)$ where $\eta = (\eta_{\beta_1}, \eta_{\beta_2}, \eta_\psi)$ and $\sigma = (\sigma_{\beta_1}, \sigma_{\beta_2}, \sigma_\psi)$. I obtain estimates for the hyperparameters and then base gene-specific inference on the posterior conditional on these estimates \citep{niemi2015empirical}.

To obtain normalization factors $\hat{\gamma}$, I use the weighted trimmed mean of $M$ values (TMM). I use edgeR to obtain genewise dispersion estimates, $\hat{\psi}_g$, and the generalized linear model methods to obtain estimates for the remaining gene-specific parameters ($\hat{\beta}_{g1}, \hat{\beta}_{g2}$)\citep{robinson2010scaling}. Using $\hat{\theta}_g = (\hat{\beta}_{g1} , \hat{\beta}_{g2}, \hat{\psi}_g)$, I estimate hyperparameters for the location and scale parameters in the hierarchical model using a central method of moments approach. 

Conditional on the estimated normalization factors $\hat{\gamma}$ and hyperparameters $\hat{\pi}$, I perform a Bayesian analysis to re-estimate the gene-specific parameters and describe their uncertainty \citep{niemi2015empirical}. Equation \ref{eq:5} shows that conditional on $\hat{\gamma}$ and $\hat{\pi}$, the gene-specific parameters are independent and therefore conditional posterior inference across the genes can be parallelized. 

\begin{equation}
\label{eq:5}
\begin{split}
& p(\theta | y, \hat{\pi}, \hat{\gamma})  \propto \\ & \prod_{g=1}^{G} \left[ \prod_{i=1}^{2} \prod_{j=1}^{n_i} NB(y_{gij} ; \exp(\lambda_{gi} + \hat{\gamma}_{ij}), \exp(\psi_g)) N(\beta_{g1} ; \hat{\eta}_{\beta_1}, \hat{\sigma}^2_{\beta_1}) p(\beta_{g2} ; \hat{\eta}_{\beta_2}, \hat{\sigma}_{\beta_2}) N(\psi_g ; \hat{\eta}_{\psi}, \hat{\sigma}^2_{\psi})  \right]
\end{split}
\end{equation}

To perform the conditional posterior inference on the gene-specific parameters, I use the statistical software Stan \citep{stan2014stan} executed through the RStan interface \citep{team2016rstan}. Stan implements a Hamiltonion Monte Carlo \citep{neal2011mcmc} to obtain samples from the posterior in equation \ref{eq:5}. 


\subsection{Gene expression differentiation}

In the maize context that motivates this work, I am interested in differential expression (DE). For a specific gene $g$, non-DE occurs when expected expression in the second variety is the same as the expected expression of first variety, i.e., $\mu_{g1} = \mu_{g2}$, or equivalently, $\beta_{g2}=0$.  I evaluate measurements based on empirical Bayes estimates of their posterior probabilities, e.g., 

\begin{equation}
\label{eq:6}
P(DE_g | y, \hat{\pi}, \hat{\gamma}) =min( P(\beta_{g2}< 0 | y, \hat{\pi}, \hat{\gamma}),  P(\beta_{g2}> 0 | y, \hat{\pi}, \hat{\gamma}))
\end{equation}

$$P(\beta_{g2}< 0 | y, \hat{\pi}, \hat{\gamma}) \approx \frac{1}{M} \sum_{m=1}^M I(\beta_{g2} ^ {(m)} < 0)$$

$$P(\beta_{g2}> 0 | y, \hat{\pi}, \hat{\gamma}) \approx \frac{1}{M} \sum_{m=1}^M I(\beta_{g2} ^ {(m)} > 0) $$

where $\beta_{g2}^{(m)}$ is the $m^{th}$ MCMC sample from the empirical Bayes posterior.



I construct a ranked list of genes according to the minimum of $P(\beta_{g2}< 0 | y, \hat{\pi}, \hat{\gamma})$ and $P(\beta_{g2}> 0 | y, \hat{\pi}, \hat{\gamma})$. Geneticists can use this list to prioritize future experiments to understand the molecular genetic mechanisms for differential expression \citep{niemi2015empirical}. 

I will use the term ebayes to refer to the approach defined in Sections 2.1 - 2.2 and I am assuming normal distribution for half-variety differences.

\section{Alternative DGE tools}

I compared the ebayes method to five alternative methods. To follow the recent progress in the RNA-Seq DE area, I selected two widely used methods, {\tt edgeR, DESeq}, and three other newly released DE analysis packages {\tt DESeq2, EBSeq, and sSeq}. For each method, I attempted to provide a measure of the strength of DE for each gene such that small values of this measure indicate support for DE. 


{\tt edgeR} is designed for the analysis of replicated count based expresison data and is an implementation of methodology developed by Robinson and Smyth \citep{robinson2007moderated}. {\tt edgeR} determines DE using empirical Bayes estimation and exact tests based on a negative binomial model. In particular, an empirical Bayes procedure is used to moderate the degree of overdispersion across genes by borrowing information between genes.An exact test analogous to Fisher's exact test but adapted to overdispersed data is used to assess DE for each gene. As default, the TMM normalization procedure is carried out to account for the different sequencing depths between the samples, wheareas the Benjamini-Hochberg procedure is used to control the FDR\citep{seyednasrollah2013comparison}. To construct a measure of DE, I computed the maximum likelihood estimates of the $\mu_{gi}$ parameters for all genes using edgeR's built-in Fisher scoring algorithm, and then used exact tests to calculate p-value for each gene $p_g$ for testing $H_{g0}: \mu_{g1} = \mu_{g2}$, which are adjusted for multiplicities using Benjamini-Hochberg. I use $p_g$ as a measure negatively associated with strength of evidence for DE. {\tt edgeR} moderates the dispersion estimate for each gene toward a local estimate from genes with similar expression strength, using a weighted conditional strength.

{\tt DESeq} also models the count of reads with the Negative Binomial distribution, but modeles the observed relationship between mean and variance when estimating dispersion, allowing a more general, data-driven parameter estimation\citep{seyednasrollah2013comparison}. It has the same hypothesis statement and exact test as {\tt edgeR}. {\tt DESeq} detects and corrects dispersion estimates that are too low through modeling of the dependence of the dispersion on the average expression strength over all samples. 

{\tt DESeq2} is a successor to {\tt DESeq}. {\tt DESeq2} sequentially estimates a prior distribution for the true dispersion values around the fit, then provide the maximum a posteriori (MAP) as the final estimate. It differs from {\tt DESeq}, which used the maximum of the fitted curve and the gene-wise dispersion estimate as the final estimate. It also differs from {\tt edgeR}, as {\tt DESeq2} estimates the width of the prior distribution from the data and therefore automatically controls the amount of shrinkage based on the observed properties of the data. In contrast, {\tt edgeR} require a user-adjustable parameter, the prior degrees of freedom, which weighs the contribution of the individual gene estimate and dispersion fit\citep{love2014moderated}. 


{\tt sSeq} can be used to test for differential expression between any two varieties based on the shrinkage estimation of dispersion in Negative Binomial models\citep{yu2013sseq}.The model has little practical difference from the model in {\tt DESeq}. Yu and Huber use the Hansen's generalized shrinkage estimator $\hat{\phi}_g$ in conjunction with the NB distribution to test genes for differential expression. They follow {\tt edgeR, DESeq} by testing $H_{g0}: \mu_{g1} = \mu_{g2}$ per gene with the exact test. Under $H_{g0}$, the p-values are calculated with respect to $Y_{gi.} \stackrel{H_{g0}}{\sim} NB(\sum_{j} s_{ij}\mu_{g}, \phi_{g}/\sum_{j} s_{ij})$ and are adjusted to control the false discovery rate\citep{yu2013sseq}. 

{\tt EBSeq} provides posterior probabilities as the evidence in favor of DE. Estimates of the gene-specific means and variances are obtained via method-of-moments, and the hyperparameters are obtained via the expectation-maximization (EM) algorithm\citep{leng2013ebseq}. {\tt EBSeq} estimates the posterior likelihoods of differential expression by the aid of empirical Bayesian methods. To account for the different sequencing depths, a median normalization procedure similar to {\tt DESeq} is used. 