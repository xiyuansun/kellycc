\chapter{Simulation Study}

\section{Simulation Study based on a maize experiment}

We built a simulation framework that aims to reflect the reality of RNA-seq data. 

\subsection{Parameter Estimation}

To assess the efficacy of eBayes method to identify DE genes, we used a maize dataset with varieties $B73$ and $Mo17$ \citep{paschold2012complementation} to determine realistic parameter values for our simulation study. Chapter 2 describes the maize dataset in detail. 

We removed the genes with zero counts in all conditions, as well as genes whose maximum counts are less than $5$ as recommended \citep{rau2013data}.The description of parameters for the real RNA-seq dataset is summarized in Table \ref{tab:Parameter-Estimation}. 

\begin{table}[ht]
\centering
\begin{tabular}{|r|r|}
\hline
Number of Trimmed Genes & 27619 \\ 
\hline
Number of Samples & 8\\
\hline
  Median expression ($\log{2}$ counts per million) & 3.92 \\ 
  \hline
  Median dispersion & 0.03 \\ 
  \hline
  Median $\log{2}$ fold change (LFC) of genes & 0.39 \\ 
  \hline
  Median library size (sum of total counts, $\log{10}$) & 6.99 \\ 
   \hline
\end{tabular}

\caption{Description of estimated parameters}
\label{tab:Parameter-Estimation}


\end{table}

We estimated parameters from the maize RNA-seq data \citep{paschold2012complementation}, and fit it by GLM with negative binomial distribution. The genewise dispersions for negative binomial GLMs were estimated using Cox-Reid Adjusted Profile Likelihood\citep{mccarthy2012differential}. This method modifies the maximum likelihood estimate of dispersion by accounting for the experimental design through Fisher's information matrix in the log-likelihood function\citep{mccarthy2012differential}. 

In summary, the library size (reads mapped to the transcriptome) is $\log{10}$ mean of $6.99$, the normalized median gene expressions $\log{2}$ counts per million (CPM) is $3.92$, and the median LFCs of DE genes is $0.39$. 


\subsection{Model}

In our simulated data, we used a generalized linear model (GLM) with negative binomial distribution. For the dataset of two groups, the counts for a particular gene $g$ in group $i$ replicate $j$ were modeled by equation \ref{eq:1}:

\begin{equation}
\label{eq:1}
Y_{gij} \stackrel{ind}{\sim} NB(mean = \mu_{gi}, var = \mu_{gi}(1+\mu_{gi}\phi_{g})
\end{equation}
where $\phi_g$ is the genewise dispersion calculated by the CR-APL method, and the expected value $\mu_{gi}$ sis a function of the library size of group $i$ replicate $j$ as equation \ref{eq:2}:

\begin{equation}
\label{eq:2}
\log_e \mu_{gi} = x_{gi}^T \beta_g + \log_e N_{ij}
\end{equation}
Here $\mu_{gi}$ is the expected counts of gene $g$ in group $i$, $N_{ij}$ is the normalized library size for group $i$ replicate $j$, $\beta_g$ is the vector of coefficients for the two experimental conditions (two groups), $x_{gi}$ is a vector of length $2$ indicating whether group $i$ replicate $j$ belongs to group one or group two in the experiment. The LFC was determined by the difference of the two elements of $\beta_g$. 



\subsection{Simulation Scenario}

To evaluate the performance of eBayes method and the alternative methods across a variety of reasonable scenarios, we created several options to make up each scenario. We set up the simulation scenarios as the following table \ref{tab:Scenario}:

\begin{table}[H]
\centering
\begin{tabular}{|r|r|r|r|r|}
\hline
sc & nGenes & nSamples & pDiff(\%) \\ 
\hline
1 & 10000 & 8 & 10 \\ 
\hline
2 & 10000 & 8 & 30 \\ 
\hline
3 & 10000 & 8 & 1 \\
\hline
4 & 10000 & 4 & 10 \\
\hline
5 & 10000 & 4 & 30 \\
\hline
6 & 10000 & 4 & 1 \\ 
\hline
7 & 10000 & 16 & 10 \\
\hline
8 & 10000 & 16 & 30 \\ 
\hline
9 & 10000 & 16 & 1 \\
\hline
10& 1000 & 8 & 10 \\
\hline
11 & 1000 & 8 & 30 \\
\hline
12 & 1000 & 8 & 1 \\ 
\hline
13 & 1000 & 4 & 10 \\
\hline
14 & 1000 & 4 & 30 \\
\hline
15 & 1000 & 4 & 1 \\ 
\hline
16 & 1000 & 16 & 10 \\
\hline
17 & 1000 & 16 & 30 \\ 
\hline
18 & 1000 & 16 & 1 \\ 
\hline
\end{tabular}
\caption{Simulation Scenario Table}
\label{tab:Scenario}
\end{table}


We set up a simulation framework with parameters based on the joint distribution of mean $\mu_{gi}$ and gene-wise dispersion estimates $\phi_g$ from the maize RNA-seq data \citep{paschold2012complementation}.

We generated true NB model parameters, $\mu_{gi}$ and $\phi_g$, using the joint distribution of estimates $\hat{\mu}_{gi}$ and $\hat{\phi}_g$, estimated using {\tt edgeR} from the real dataset \citep{paschold2012complementation}. The derived from real data parameters were used to simulate the counts, from a NB distribution. 


For a particular gene in the simulated dataset, the number of true DE tag was determined via the proportion of differential gene expression (pDiff) in the simulation scenario setup. I randomly selected $nGenes \times (1-pDiff)$ genes and assign the expected variety count means the same as $\mu_{g1} = \mu_{g2} = 1/(nSample)\sum_{j=1}^{nRep} \left[ N_{1j}\exp{LCF_{g,1}}+ N_{2j}\exp{LCF_{g,2}} \right]$ for the selected non-DE genes, where $LCF_{g,i}, i=1,2$ was the log of change of fold estimates got from the parameter estmation based on the real RNA-seq dataset.

We simulated $nGenes$ total number of genes with $nSample$ number of samples, among which $pDiff$ percent of them are true DE genes. For each scenario, we simulated 5 such datasets with different seeds.









